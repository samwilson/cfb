\biohead{Boyd Gilmour}{Boyd_Gilmour}{}

Boyd Gilmour was born on March 22, 1814, in Riccarton ("Gilmour Riccarton 3 April 1814 this day was baptized Boyd son of Joseph Gilmour and Mary Clark born 22 Mar."\citeref{BGbirth}) and died on 26 March 1869 (Ayrshire, Scotland).  His parents were Joseph Gilmour (\p{Joseph_Gilmour}) and Mary Boyd Clark. His siblings were Elizabeth Gilmour (1797--1870), Joseph Gilmour (1802--1851),
James Gilmour (1805--1866), Allan Gilmour (1807--1854), Andrew Gilmour (1810--1874), Robert Gilmour (1812--1841).


He married Jean Dunsmore (Dunsmuir) and they had eight children:  Jean Gilmour (1836--?), Joseph Gilmour (1838 -- bef. 1840),\citeref{JGbirth} Joseph Gilmour (1840--?), Mary Gilmour (1843--1899),  Marion Gilmour (1847--1928), Boyd Gilmour (1849--?), Allan Columbia Gilmour (1851--?), John Gilmour (1854--1856).

After her death he remarried Elizabeth Howatson.

On 19 December 1850, Boyd and his family sailed on the Pekin for Fort Vancouver, and the journey took 191 days. On 18 July 1851 they sailed to Fort Rupert, on Vancouver Island where he took up a contract to develop new coal mines for the Hudson Bay Company (the HBC had recruited expert miners and their families on three-year contracts from the Orkney Islands and the county of Ayrshire). He struggled unsuccessfully to develop a producing coal operation, (with his nephew Robert Dunsmuir, who was to become one of the richest men on the west coast) at Fort Rupert. Life at Fort Rupert was harsh. When the miners arrived they found no working mine, inferior coal, food shortages, and danger from warring native tribes. The settlement consisted of a defensive wooden surround in the traditional wild-west style, and single room log cabins with a central stone fireplace and bunk beds set against the wall. Water was drawn from a communal well: communal ovens were used for cooking. The coal there was poor, so Fort Rupert’s mines were eventually abandoned after many miners breached their contracts and fled to the California gold fields. (S6) Those few that remained moved to Fort Victoria, including Boyd and his family, on 24 August 1852 after Governor Douglas instructed them to move 200 miles south to Nanaimo, a small port which was based on the fur trade and fishing. It was here that a local Indian told the settlers where they could find stones that burn - thus a coal seam was discovered. Work proceeded but living conditions were difficult. Living conditions were only slightly better at Nanaimo and Jean Gilmour refused to live there. The Gilmours returned to Scotland in 1854, when Governor Douglas refused to increase their pay rates.

After Jean died in 1856, Boyd is shown in the 1861 Census as living in Old Hurlford and is a Coalmaster (widower,age 46) with his children Mary (18, \p{Mary_Gilmour}), Marian (14), Boyd (12), and Allan Columbia (9). He then remarried later that year (11 November) to Elizabeth Howatson, a 20 year old farmer's daughter (living at Hill Farm) and had three more children.
When his daughter Mary married Samuel West Croskery in ? his occupation was noted as Coalmaster.\citeref{Marriagecert}

In the 1868 Hurlford District Directory his properties are listed as Woodend, Burnbank, Ladyton, and Goatfoot Collieries.

On his death certificate he is listed as "Coalmaster", and died at his home, "Riverside Cottage", Loudon Parish. His father Joseph was down as a coal miner. He died from 'fatty degeneration of the heart ten days from appearance of symptoms' and the death was reported by his brother Andrew Gilmour, butcher, also of Loudon Parish. His will includes details about a contract with his son Allan, and provision is made for his second wife Elizabeth (use of his house in Titchfield Street, Galston, and a yearly annuity of (pounds) 120 until the youngest child attains the age of 21 after which the entitlements were reduced - payable Whitsunday and Martinmas. Plus reasonable assistance after his death to provide his wife and children with mourning. When or if she remarries, she would then receive (pounds) 20 per annum. She "is obliged to maintain and upbring in a manner suitable to that station such of his children who have not attained majority.")(S6 

\begin{references}
 \reference{Marriagecert}{Marriage record of Samuel Croskery and Mary Gilmour}
\reference{BGbirth}{Birth record of Boyd Gilmour, Scotlands People, 03/04/1814 O.P.R.births 611/00 0010 0319}

\reference{JGbirth}{ Scotlands people, BIrth records, Riccarton Parish, record no.611, vol.0020}

%\reference{LABEL}{Citation text.}

\end{references}
