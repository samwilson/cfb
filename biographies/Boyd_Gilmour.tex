\biohead{Boyd Gilmour}{}{}

Boyd Gilmour was born on March 22, 1814, in Riccarton, Ayrshire, Scotland. \cite{BGbirth})His parents were \bioref{Joseph_Gilmour}, a coalminer, and \bioref{Mary_Boyd_Clark}.
He had six siblings: Elizabeth Gilmour (1797--1870), Joseph Gilmour (1802--1851),James Gilmour (1805--1866), Allan Gilmour (1807--1854), Andrew Gilmour (1810--1874),
and Robert Gilmour (1812--1841).

He married \bioref{Jean_Dunsmore} (also spelled as Dunsmuir) and they had eight children:
Jean Gilmour (1836--?), Joseph Gilmour (1838 -- bef. 1840), Joseph Gilmour (1840--?), \bioref{Mary_Gilmour},
Marion Gilmour (1847--1928), Boyd Gilmour (1849--?), Allan Columbia Gilmour (1851--?), and John Gilmour (1854--1856).

On 19 December 1850, Boyd and his family sailed on the ship Pekin for Fort Vancouver, and the journey took 191 days. On 18 July 1851 they sailed to Fort Rupert, on Vancouver Island where he took up a contract to develop new coal mines for the Hudson Bay Company (the HBC had recruited expert miners and their families on three-year contracts from the Orkney Islands and the county of Ayrshire). He struggled unsuccessfully to develop a producing coal operation, (with his nephew Robert Dunsmuir, who was to become one of the richest men on the west coast) at Fort Rupert. Life at Fort Rupert was harsh. When the miners arrived they found no working mine, inferior coal, food shortages, and danger from warring native tribes. The settlement consisted of a defensive wooden surround in the traditional wild-west style, and single room log cabins with a central stone fireplace and bunk beds set against the wall. Water was drawn from a communal well: communal ovens were used for cooking. The coal there was poor, so the Fort Rupert mine was eventually abandoned after many miners breached their contracts and fled to the California gold fields. Those few that remained moved to Fort Victoria, including Boyd and his family, on 24 August 1852, after Governor Douglas instructed them to move 200 miles south to Nanaimo, a small port which was based on the fur trade and fishing. It was here that a local Indian told the settlers where they could find stones that burn - thus a coal seam was discovered. Work proceeded but living conditions were difficult. Living conditions were only slightly better at Nanaimo and Jean Gilmour refused to live there. The Gilmours returned to Scotland in 1854, when Governor Douglas refused to increase their pay rates. \cite{BGilmourCanada}

After Jean died in 1856, Boyd is shown in the 1861 Census as living in Old Hurlford and is a Coalmaster (widower, aged 46) with his children Mary, Marian (14), Boyd (12), and Allan Columbia (9). He then remarried later that year (on 11 November 1861) to Elizabeth Howatson, a 20 year old farmer's daughter (then living at Hill Farm) and had three more children: Elizabeth Gimour (1862--?), Janet Gilmour (18634--?) and Andrew Gilmour (1866--?).

When his daughter Mary married Samuel West Croskery in August 1874, his occupation was noted as having been a Coalmaster.\cite{SMWCmarriage}
Boyd was prominent in the town of Galston: ``By the last quarter of the 19th century, Galston was best described as a mining town. Mr Boyd Gilmour, Coalmaster, after whom Boyd Street was named, had been one of the sponsors of the new burgh, and the largest number of male inhabitants were miners.''\cite{BGilmourGalston}

In the 1868 Hurlford District Directory his properties are listed as Woodend, Burnbank, Ladyton, and Goatfoot Collieries.

Boyd died on 26 March 1869 at Loudon, Ayrshire. \cite{BGilmourDeath} On his death certificate he is listed as `Coalmaster', and died at his home, \emph{Riverside Cottage}, Loudon Parish. His obituary in the Kilmarnock Standard, 3 April 1869, read:\cite{BGobituary}

\begin{quotation}
Boyd Gilmour of Riverside Cottage, Galston. He was Coalmaster of the firm Boyd Gilmour and Co., Burnbank, Ladyston and Goatfoot Collieries. He served as magistrate of the Burgh in Galston. It is our painful duty to record the decease of one of our most respected and enterprising townsmen, Mr Boyd Gilmour, Coalmaster, who died on Friday night last in the 54th year of his age.
\end{quotation}

He died from `fatty degeneration of the heart ten days from appearance of symptoms' and the death was reported by his brother Andrew Gilmour, butcher, also of Loudon Parish. His will includes details about a contract with his son Allan, and provision is made for his second wife Elizabeth (use of his house in Titchfield Street, Galston, and a yearly annuity of (pounds) 120 until the youngest child attains the age of 21 after which the entitlements were reduced---payable Whitsunday and Martinmas. Plus reasonable assistance after his death to provide his wife and children with mourning. When or if she remarries, she would then receive (pounds) 20 per annum. She ``is obliged to maintain and upbring in a manner suitable to that station such of his children who have not attained majority.''

His will read as follows:

\begin{quotation}
Inventory of the Will of BOYD GILMOUR

Ayr the eighth day of May 1869 J and J Hendrie Solicitors in Galston who produced inventory of the personal estate of the deceased Boyd Gilmour designated also General Trust Dispersion and Settlement by the deceased and of which inventory follows.

Inventory of the personal estate of the deceased wheresoever situated of Boyd Gilmour Coalmaster residing in Galston who died there on the twenty sixth day of March 1869.

Scotland, Personal Property:\hfill(pounds/shilling/pence)

\begin{enumerate}
\item Cash in the house\hfill---
\item Household furniture and other effects in the deceased house conform to appraisement\hfill1257 – 6 – 0
\item Amount at credit of deceased with the firm of Boyd Gimour and company Coalmasters of which deceased was a partner including of his share of the stock in trade, machinery, offices and office furniture conforming to the books of said firm\hfill2096--7--0
\item Amount at credit of deceased with the Maryport Iron Company of which deceased was partner confirm to the books of said Company\hfill1154--15--5
\item Principal sum contained in a Policy of Assurance no. 4898 granted by the Scottish National Insurance Company on the life of the deceased dated 22 March 1867\hfill500--0--0
\item Rents of heritage due by the following tenants falling under executary:
\begin{enumerate}
 \item Archibald Falconer (\textonehalf yr)\hfill2.10.0
 \item Joseph Gilmour (\textonehalf yr)\hfill2.10.0 5 – 0 – 0
\end{enumerate}
\item Amount of personal estate in Scotland\hfill3881--9--11
\end{enumerate}

Witness and executors to the above will were Allan Gilmour, Coalmaster residing at Woodend near Kilmarnock, along with John Gilmour Coalmaster residing at Hillhead Villa, Kilmarnock, also James Hendrie, solicitor Galston and John Maclatchy Doctor of Medicine residing at Woodend Cottage near Kilmarnock.
\end{quotation}
