\biohead{Charles Frederick Strangeways Barker}{}

He was born on 21 August 1878 in	Liverpool, Lancashire to Thomas Henry Barker (\p{Thomas_Henry_Barker}) and Mary Ellen Moulsdale (\p{Mary_Ellen_Moulsdale}), and christened on 30 September 1877 at St Brides, Liverpool \cite{CFSBarkerBaptism}. He had six siblings:  James Denton Barker (\p{James_Denton_Barker}), Reverend Thomas Percy Conyers Barker (\p{Thomas_Percy_Conyers_Barker}), Francis Darcy Mead Barker (\p{Francis_Darcy_Mead_Barker}), William Danby Holt Barker (\p{William_Danby_Holt_Barker}), Jonathan Tong Barker (\p{Jonathan_Tong_Barker}) and Henry Bertram Mitford Barker (\p{Henry_Bertram_Mitford_Barker}). 


In 1901 he was an Assistant Clerk at the Liverpool  Chamber of Commerce \cite{CFSBarkerOccupation}.  By 1910, he had enlisted in the  4th Battalion, Cheshire Regiment, Reg. No. 1021 \cite{CFSBarkerMilitary}.

He married Phyllis May Wickham and they had one daughter, Peggy. In 1930 they were living at 'Charlton', Aughton (near Ormskirk), Lancashire \cite{CFSBarker1930}.

On 18 February 1930 he was (possibly) filing for bankruptcy as an Asbestos merchant in Liverpool:

" Barker Charles Frederick Strangways of Charlton, Quarry Drive, Aughton, Ormskirk, in the county of Lancaster, ASBESTOS MERCHANT and lately carrying on business at 51 Old Hall-street in the city of Liverpool.
Court - Liverpool.
No of matter - 80 of 1921
Last day for receiving proofs March 4 1930
Name of trustee and address - Allcorn James, Government Buildings, Victoria St. Liverpool Official receiver" \cite{CFSBarker1930}


He died on 21 January 1962 at the Newsham General Hospital, Liverpool \cite{CFSBarkerDeath} and the Probate notice read:    
"Barker Charles Frederick Strangways of 365 Park Road Liverpool 8 died 21 January 1962 at Newsham General Hospital Liverpool 6. Administration Liverpool 30 March to Phyllis May Barker widow Effects (pounds)656.11s.3d.."
