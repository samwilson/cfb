\biohead{Ralph Munday}{Munday, Ralph}{c.~1910\cite{RalphMundayInUniform}}
% Q43807196

Ralph Munday was born on 26 November 1885 in Surbiton, Surrey\cite{RMundayBirth}, the only son of \bioref{John_Hill_Munday} and \bioref{Catherine_Aldridge}.
He had four siblings: \bioref{Nora_Katie_Munday}, \bioref{Kathleen_Munday}, \bioref{Mildred_Mary_Munday}, and \bioref{Margery_Munday}.

He was educated at Bilton Grange Preparatory School, Crowthorne, Berkshire and Wellington College \cite{RMundayEducation}.

He was a Solicitors Clerk in 1911\cite{RMundayOccupation}.

During the war, he served with the 9th Battalion, Sherwood Foresters, Nottingham and Derbyshire Regiment. In 1915 he underwent training in Cairo and was appointed to a temporary commission on 27 October 1915 \cite{RMundayWar} as recorded in the London Gazette:  "Appointed to a commission in 9th Battalion Notts and Derby Regt. (Sherwood Foresters) Authority General Ord. Force in Egypt No 701, 23 October 1915".  He later became a Captain. He  served at Gallipoli and on the Western Front in the First World War. He was awarded the Military Cross on 3 June 1918. He was then with a POW convoy and was demobilised on 22 August 1919.

A letter that he wrote to his sister Kathleen (held by a family member), in 1915, reads as follows:

\begin{quotation}
\begin{raggedleft}
9th Sherwood Foresters
33rd Brigade
11th Division 17.11.15
\end{raggedleft}

Dear Kath,

At last my letters can truthfully be marked on "active" service. We landed here on Sunday last (Nov 14th) ``we'' means I and a batch of other subalterns, including my friend Egerton who is in the same Battalion as I. In case you do not understand, in the Infantry there are no longer regiments as such, but for the purpose of distinction and for sentimental reasons, the names are retained and the various battalions are called after their regimental names with a number added. Perhaps you would explain this to them at home, as in writing one is apt to use the word regiment and battalion as synonomous, and they may not understand as you may be able to gather from my letter home. Am writing from a quite comfortable dug out just behind the front line trenches. We are this week, in support; which means that we do not go into the firing line trenches, but have constant fatigues to do most of the day and night. We actually go a bit further than the front line trenches sometimes. We are more or less under fire always but at most times one is pretty safe and until bullets begin to ping and whine on the ground within a yard or so one does not notice the fire much. I have not been under really heavy fire yet, I gather that that is a little unpleasant while it lasts, but if it does not go on too long from my impressions at present, I gather that one very soon recovers ones equinamity. From our position on the side of a hill we can see the sea and a good deal of our ? and the enemy's trenches on both flanks. We are able to watch bombardments both from our own guns and those of the enemy; in the matter of gun fire, our side does the most. Yesterday afternoon and the day before there was a tremendous shelling of the enemy positions by our chaps and land batteries. It is a curious game, neither side sees much of the other side and the only people who do any damage (except when one side is making an attack) are the snipers and the guns. The country where we are is very rocky and covered in low scrubby bushes, which cover the rocky gullies which seam all this part of the country; so the snipers get very good cover for their work. The guns as a rule do not do much damage, anyway the enemy's guns do not and I hope ours are more effective. The two things which are most troublesome are dust and flies. The latter are not as bad as those of Egypt. We had a raging thunderstorm the night before last which not only laid the dust but made most unpleasant pools in communication trenches. But the ground has dried up since then and we have plenty of dust again. Water, of course, is not very plentiful but we get enough to wash and shave in and for tea, but it mostly has to be brought up by hand from behind the line, so we cannot be wasteful with it. While we are out here we are not as well off for pay as we got there. We were drawing what was called Colonial Allowances for all sorts of strange things such as light, food, housing and c. Of course we had much bigger expenses there, than we have here. Food is about the only thing to spend ones money on here. We came from Alexandria on the sister ship to the Royal Edward that was sunk in less than five minutes, some months ago. We had some most excellent food on board, and we picked up a couple of boats with their crews from a cargo steamer which had been sunk by a German submarine ahead of us; they were glad to meet us; but they had been given time to get some provisions and water into the boats and they had brought their dog January. They had a monkey on board which they took off several times, and each time he escaped and jumped back on board and the poor chap went down with the ship.

We are shifting tomorrow I think into the firing line proper. It is not a very dangerous place as a rule. It is not such dangerous work as we have been on as a matter of fact. We shall probably be there for about a week and then we shall probably move further back. We get a better time than the men while we are in the firing line as we have dug outs to get into when we are not on duty.

Well up to the present have managed to keep healthy and cheerful. There is no need for anyone to worry about me.

Hope you, Mead and Jim are all well.

Love, your affect. brother

Ralph Munday
\end{quotation}

A postcard from Ralph's father, John Hill Munday, to his sister Catherine (Mrs James Denton Barker), written on 3 January 1918 (held by a family member) reads:

\begin{quotation}
In `The Times' of 22 December Ralph is given as ``mentioned'' in Sir Douglas Haig's dispatch in the list of Nottinghamshire and Derbyshire Regiment but we all overlooked it till mentioned in letter rec'd from Ralph last night.

Mother, glad to say, continues better.

Hope you are not paralysed with the cold as I am.

Much love JMH, Cedar Lodge, 21 St Johns Road, Putney Hill, SW15.
Telephone P.O. 463, Putney.
\end{quotation}

Before the War Ralph had worked as an articled clerk for his father but in 1919, following his father's death, he did not return to work in the law but emigrated to Western Australia.  There he managed a business related to the motor trade, but that did not prosper and he worked for a while on a farm on the Wheatbelt. Then he went to Java,  where he managed a export company owned by brother-in-law Charles Hadden.  He married \bioref{Vera_Maunder} in Java,\cite{LadiesSection} and they adopted one daughter, \bioref{Julia_Leat}.
They moved back to Western Australia in 1939 and lived on Forrest Street, Cottesloe.

During the Second World War he worked in the Ministry of Munitions in Perth, and later, in retirement, he had a strong interest in growing orchids and kept tropical fish.

He died in 1962.\cite{RalphMundayBMD}
