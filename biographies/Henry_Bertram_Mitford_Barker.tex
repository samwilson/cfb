\biohead{Henry Bertram Mitford Barker}{}

Henry (known as Harry) was born on 8 March 1886 in Liverpool [HBMBarkerBirth],  to Thomas Henry Barker \bioref{Thomas_Henry_Barker} and Mary Ellen Moulsdale \bioref{Mary_Ellen_Moulsdale}\cite{HBMBarkerBirth}. He had six brothers,  James Denton Barker \bioref{James_Denton_Barker}, Charles Frederick Strangways Barker \bioref{Charles_Frederick_Strangeways_Barker}, Thomas Percy Conyers Barker \bioref{Thomas_Percy_Conyers_Barker}, Francis Darcy Mead Barker \bioref{Francis_Darcy_Mead_Barker}, William Danby Holt Barker \bioref{William_Danby_Holt_Barker}, and Jonathan Tong Barker \bioref{Jonathan_Tong_Barker}.

He was educated at Liverpool College (1905--1916) and lived with his family at "Ormesby", West Kirby, Cheshire.  At the outbreak of World War One, he was employed as a grain merchant.  He then joined the army as a lance corporal in the Yeoman Infantry and  was part of the British Expeditionary Force, France from 1 November 1914 to 20 April 1916. He was then transferred and promoted to Second Lieutenant with the 3/9th King's Liverpool Regiment on 22 April 1916 and returned to France.  He was wounded on 31 July 1917 at Ypres,  and admitted to the Prince of Wales Hospital on 10 August 1917, where he  was subsequently listed as 'unfit for service'.  A letter he wrote later that year reads as follows:
 \quotation{
                                                                           Kings Lancashire Hospital
                                                                          Imperial Hydro
                                                                          St Annes-on-Sea
                                                                          2/12/17
 To the Secretary of the War Office
 F3 Department, London
 
 Sir,
 I have the honour to request that you will grant me a wound gratuity.  I was wounded on the 31st July, 17, at Ypres, by a bullet passing through the thigh.  The femoral vein was injured which has interfered with the circulation in the leg, causing swelling and weakness.  I have the honour to be
 Yr. obedient servant
 Henry Bertram Mitford Barker 2nd Lieutenant, 1/9th King's Liverpool Regiment.
 \endquotation
 \cite{HBMBarkerWarrecord}
 
 He was granted Service Medals for his service \cite{HBMBarkerMedal} in the war.
